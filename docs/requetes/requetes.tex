%% LyX 2.3.2-1 created this file.  For more info, see http://www.lyx.org/.
%% Do not edit unless you really know what you are doing.
\documentclass{article}
\usepackage[T1]{fontenc}
\usepackage[cp1251]{inputenc}
\usepackage{amssymb}
\usepackage[belarusian]{babel}
\begin{document}

\section*{Modifications}
\begin{itemize}
	\item Pour la d\'{e}normalisation : Toutes les relations $1..n-1..1$ peuvent
		\^{e}tre simplifier en ajoutant dans la table $1..1$ une clef pour
		acc\'{e}der aux \'{e}l\'{e}ments de la table $1..n$. (Bonnes pratiques).
		On peut donc ajouter \`{a} <<Impot>>, \`{a} <<Papier>>, <<ElementsJuridique>>,
		et <<AidesSociales>> une clefs \'{e}trang\`{e}re vers <<IdResident>>
		repr\'{e}sentant le r\'{e}sident auquel l'Impot, le Papier, l'aide
		sociales ou l'\'{e}l\'{e}ment juridique est associ\'{e}.

	\item Pour les tables <<Papier>>, <<Impot>> et <<AidesScociales>>:
		ajout d'un \'{e}l\'{e}ment <<etat>> qui prendra les valeurs <<EnCoursDeValidation>>,
		<<Valid\'{e}>>, <<Refus\'{e}>> ou <<Expir\'{e}>>. Cela va permettre
		pour faire une demande de Papier, Aides sociales ou une de d\'{e}claration
		d'impot, tr\`{e}s simplement. Le R\'{e}sident rempli un formulaire
		et le soumet sur la plateforme. La requ\^{e}te derriere sera simplement
		un insert dans la table voulu avec tous les \'{e}l\'{e}ments indiqu\'{e}s
		par le r\'{e}sident, seulement, l' <<etat>> du Papier, de l'Aide
		Sociale ou de l'impot cr\'{e}e sera <<EnCoursDeValidation>>, pour
		passer \`{a} <<Valid\'{e}>> ou <<Refus\'{e}>> il faudrat qu'un
		administrateur la valide. Cela permet de supprimer facilement les
		relations ternaires non souhait\'{e}.Cela permet aussi de garder pendant
		un temps souhait\'{e} les diff\'{e}rents \'{e}l\'{e}ments du dossier
		d'un R\'{e}sident m\^{e}me si ceu ci sont expir\'{e} ou sont des demandes
		refus\'{e} (on peut imaginer qu'apr\`{e}s un temps ils sont supprim\'{e}s
		de la base de donn\'{e}es. Ainsi par exemple si un R\'{e}sident veut
		faie une demande de Care d'identit\'{e}, il n'a qu'a remplir le formulaire
		en indiquant <<carte d'ident\'{e}>> en nom ainsi que les autres
		information utile, il en d\'{e}coulera un insert dans la table <<Papier>>
		avec la cr\'{e}ation du papier demand\'{e} associ\'{e} au formulaire,
		seulement l'\'{e}tat du papier sera <<EnCoursDeValidation>> jusqu'\`{a}
		ce qu'un administrateur le passe en <<Valid\'{e}>> ou <<Refus\'{e}>>.

	\item Pour la table <<ElementsJudicaires>> : ajout d'un \'{e}l\'{e}ment
		<<peine>> qui pourrat prendre les valeur <<EnCoursDExecution>>,
		<<Execute>>, <<NonEligilible>>. La valeur <<NonEligible>> sera
		prise s'il n'y a pas lieu d'avoir une peine pour l'\'{e}l\'{e}ment
		du casier. Les valeurs <<EnCoursDExecution>> et <<Execute>> parle
		d'elles m\^{e}me et permettront ainsi un meilleure suivi de la personne
		et surtout de pouvoir facilement faire la requ\^{e}te <<Est ce qu'un
		r\'{e}sident purge actuellement une peine?>> puisqu'alors il suffit
		juste de s\'{e}lectionner les ElementsJudiciaires correspondant \`{a}
		ce r\'{e}sident puis de egarder les valeurs de l'\'{e}l\'{e}ment <<peine>>
		pour chacun d'entre, si au moins l'un de \`{a} la valeur <<EnCoursDExecution>>
		la r\'{e}ponse sera oui.
\end{itemize}

\section*{Requ\^{e}tes}
	\begin{itemize}
		\item Liste des aides sociales que per\c{c}oit un r\'{e}sident <<X>>.
		\begin{itemize}
			\item Reformultion : Liste de toutes les aides O\`{U} l'IdR\'{e}sident = <IdX>
			\item Alg\`{e}bre relationelle : $\sigma_{IdResident=<IdX>}AidesSociales$
			\item MySQL: SELECT {*} FROM AideSociales WHERE IdResident = <IdX>;
		\end{itemize}
		\item Combien a pay\'{e} un r\'{e}sident en imp\^{o}t depuis 10 ans
		\begin{itemize}
			\item Reformulation : Liste des montants des imp\^{o}ts O\`{U} IdR\'{e}sident
			= <IdX> ET O\`{U} la date est sup\'{e}rieur ou \'{e}gale \`{a} l'ann\'{e}e
			en cours -10. (On fait la somme des montant en dehors d'une requ\^{e}te
			MySQL).
			\item Alg\`{e}bre relationnelle : $\pi_{montant}(\sigma_{IdResident=<IdX>}(\sigma_{date\geqq<anneeEnCours>-10}Impots))$
			\item MySQL : SELECT montant FROM Impots WHERE IdResident = <IdX> AND date
			= <AnneEnCours> - 10;
			\begin{itemize}
				\item Il n'y a plus qu'\`{a} faire une boucle sur les montants obtenus pour
				les sommer
			\end{itemize}
		\end{itemize}
		\item Est-ce que la demande d'allocation ch\^{o}mage a \'{e}t\'{e} valid\'{e}
		pour un r\'{e}sident donn\'{e}.
		\begin{itemize}
			\item Reformulation : Etat de de l'Aides Sociales O\`{U} IdResident = <IdX>
			ET O\`{U} nom = <Chomage>
			\item Alg\`{e}bre relationnelle : $\pi_{etat}(\sigma_{IdResident=<IdX>}(\sigma_{nom='Chomage'}AidesSociales))$
			\item MySQL : SELECT etat FROM AidesSociales WHERE IdResident = <IdX> AND
			nom = 'Chomage';
		\end{itemize}
		\item Est ce que le r\'{e}sident purge actuellement une peine ?
		\begin{itemize}
			\item Reformulation : Liste des Elements du Casier Judiciaire O\`{U} IdResident
			= <IdX> ET O\`{U} peine = 'EnCoursDExecution' ( Si liste vide alors
			r\'{e}ponse non sinon r\'{e}ponse oui)
			\item Alg\`{e}bre relationnelle : $\sigma_{IdResident=<IdX>}(\sigma_{peine='EnCoursDExecution}ElementsCasierJudicaire)$
			\item MySQL : SELECT {*} FROM ElementsCasiersJudiciaire WHERE IdResident
			= <IdX> AND peine = 'EnCoursDExecution';
			\begin{itemize}
				\item (Possibilit\'{e} de faire une projection sur une peine si on veut
				simplement avoir la liste des peines 'EnCoursDExecution' cela d\'{e}pend
				de l'usage choisit. Afficher tous les \'{e}l\'{e}ments si la peine
				est en cours d'\'{e}x\'{e}cution \`{a} l'avantage de permettre 'en
				savoir plus sur la dite peine).
			\end{itemize}
		\end{itemize}
		\item Information sur l'extrait d'\'{e}tat civile d'un r\'{e}sident <<X>>
		\begin{itemize}
			\item Reformulation : Liste des attributs de Papier O\`{U} IdResident=<IdX>
			ET O\`{U} type=EtatCivile
			\item Alg\`{e}bre relationne : $\sigma_{IdeResident=<IdX>}(\sigma_{type='EtatCivile'}Papier)$
			\item MySQL : SELECT {*} FROM Papier WHERE IdResident =<IdX> AND type =
			'EtatCivile';
		\end{itemize}
		\item Effectuer une demande de renouvellement d'un passeport
		\begin{itemize}
			\item Pr\'{e}requis : Le R\'{e}sident rempli un formulaire puis le soumet,
			les valeurs qu'il aurat renseign\'{e} seront mises ap\`{e}s v\'{e}rifications
			dans les variables : <<IdDemande>>, <<IdResident>>, <<typeDemande>>,
			<<nomDemande>>, <<DateDebutDemande>>, <<DateFinDemande>>. Le
			r\'{e}sident n'aurat en soit qu'a renseign\'{e} le type de demande
			qu'il souhaite faire ici <<Passeport>> le reste peut \^{e}tre obtenu
			sans son intervention. Dans tous les cas, ces informations doivent
			\^{e}tre disponible avant toute requ\^{e}te.
			\item Reformulation : Cr\'{e}er un nouveau Papier qui a pour valeur les
			valeurs des variables r\'{e}cup\'{e}r\'{e}.
			\item MySQL : INSERTE INTO Papier (ID, IdResident, type, nom, DateDebut,
			DateFin ) VALUES (IdDemande, IdResident, typeDemande, nomDemande,
			DateDebutDemande, DateFinDemande);
		\end{itemize}
	\end{itemize}

\end{document}
